\documentclass{article}
\usepackage{graphicx} % Required for inserting images
\usepackage{amsmath}
\usepackage{amsfonts}


\title{Introducci\'on al lenguaje de las matem\'aticas}
\author{Roberto G. Fernández}
\date{Febrero 2024}

\begin{document}

\maketitle


\section{Conjuntos num\'ericos}
\subsection{Teor\'ia de conjuntos}

\subsubsection{Conceptos b\'asicos}
Un conjunto es una colecci\'on de objetos, y est\'a bien
definido si se sabe si un determinado elemento pertenece o no
a \'el.

Llamaremos elemento a cada uno de los objetos que forman parte de un conjunto. Usualmente se representan con una letra min\'uscula, en tanto que los conjuntos, con una may\'uscula.
Para indicar que un elemento x pertenece a un conjunto A, escribimos:

\begin{equation}
x \in A
\end{equation}

La negaci\'on de \begin{math}x \in A\end{math} se escribe:

\begin{equation}
x \notin A
\end{equation}

Dos conjuntos especiales son:
\begin{itemize}
  \item El conjunto universal, que representaremos con la letra U, es el conjunto de todas las cosas sobre las que estemos tratando.
  \item El conjunto vac\'ıo, que se denota \begin{math}\emptyset\end{math}, que no tiene elementos.
\end{itemize}


\subsection{Conjuntos de n\'umeros}

\subsubsection{N\'umeros naturales}

El conjunto de n\'umeros naturales, al que notamos con $\mathbb{N}$, es el que usamos usualmente para contar, es decir:

\begin{equation}
\mathbb{R} = {1, 2, 3, 4, 5, .....}
\end{equation}

Propiedades: El conjunto $\mathbb{N}$ es infinito, es ordenado, tiene primer elemento pero no \'ultimo, y es discreto (entre dos n\'umeros naturales existe un n\'umero finito de n\'umeros naturales).

\subsubsection{N\'umeros enteros}
El conjunto de n\'umeros enteros, al que notamos con $\mathbb{N}$, es el conjunto de n\'umeros naturales, agregando el cero y los enteros negativos, es decir:

\begin{equation}
\mathbb{Z} = {......, −5, −4, −3, −2, −1, 0, 1, 2, 3, 4, 5, .....}
\end{equation}

Observaci\'on: Se puede ver que $\mathbb{N} \subset \mathbb{Z}$.

Propiedades: El conjunto $\mathbb{N}$ es infinito, es ordenado, no tiene primer ni \'ultimo elemento, y es discreto (entre dos n\'umeros enteros existe un n\'umero finito de n\'umeros enteros).

\subsubsection{N\'umeros racionales}
El conjunto de n\'umeros racionales, al que notamos con $\mathbb{Q}$, es el conjunto de todos los n\'umeros que pueden expresarse como una raz\'on entre dos n\'umeros enteros (todos los n\'umeros que pueden expresarse como fracci\'on). Es decir:

\begin{equation}
\mathbb{Q} = {\frac{a}{b} / a, b \in \mathbb{Z}, b \neq 0}
\end{equation}

Observaciones:
\begin{itemize}
  \item Todo n\'umero entero es un n\'umero racional, pues puede escribirse como fracci\'on de denominador 1. Por lo tanto: $\mathbb{Z} \subset \mathbb{Q}$.
  \item Todo n\'umero fraccionario tiene una expresi\'on decimal finita o infinita peri\'odica, y rec\'ıprocamente, toda expresi\'on decimal finita o infinita peri\'odica se puede expresar como un n\'umero fraccionario.
\end{itemize}


Propiedades: El conjunto $\mathbb{Q}$ es infinito, es ordenado, no tiene primer ni \'ultimo elemento, y es denso (entre dos n\'umeros racionales existe un n\'umero infinito de n\'umeros racionales).

\subsubsection{N\'umeros irracionales}
El conjunto de n\'umeros irracionales, al que notamos $\mathbb{I}$, es el conjunto de n\'umeros que no se pueden escribir como una raz\'on entre dos n\'umeros enteros.
Observaciones:
\begin{itemize}
  \item De la definici\'on podemos concluir que el conjunto de los n\'umeros racionales y el conjunto de los n\'umeros irracionales son conjuntos disjuntos, es decir, $\mathbb{Q} \cap \mathbb{I} = \emptyset$.
  \item Todo n\'umero fraccionario tiene una expresi\'on decimal finita o infinita peri\'odica, por lo tanto, todo n\'umero irracional tiene un n\'umero infinito de d\'ıgitos no peri\'odicos.
  \item Son n\'umeros irracionales los resultados de aplicar ra\'ıces (de \'ındice par a n\'umeros naturales \'o de \'ındice impar a n\'umeros enteros) que no dan como resultado un n\'umero natural o entero.
\end{itemize}







\end{document}